% Options for packages loaded elsewhere
\PassOptionsToPackage{unicode}{hyperref}
\PassOptionsToPackage{hyphens}{url}
%
\documentclass[
]{article}
\usepackage{amsmath,amssymb}
\usepackage{lmodern}
\usepackage{iftex}
\ifPDFTeX
  \usepackage[T1]{fontenc}
  \usepackage[utf8]{inputenc}
  \usepackage{textcomp} % provide euro and other symbols
\else % if luatex or xetex
  \usepackage{unicode-math}
  \defaultfontfeatures{Scale=MatchLowercase}
  \defaultfontfeatures[\rmfamily]{Ligatures=TeX,Scale=1}
\fi
% Use upquote if available, for straight quotes in verbatim environments
\IfFileExists{upquote.sty}{\usepackage{upquote}}{}
\IfFileExists{microtype.sty}{% use microtype if available
  \usepackage[]{microtype}
  \UseMicrotypeSet[protrusion]{basicmath} % disable protrusion for tt fonts
}{}
\makeatletter
\@ifundefined{KOMAClassName}{% if non-KOMA class
  \IfFileExists{parskip.sty}{%
    \usepackage{parskip}
  }{% else
    \setlength{\parindent}{0pt}
    \setlength{\parskip}{6pt plus 2pt minus 1pt}}
}{% if KOMA class
  \KOMAoptions{parskip=half}}
\makeatother
\usepackage{xcolor}
\IfFileExists{xurl.sty}{\usepackage{xurl}}{} % add URL line breaks if available
\IfFileExists{bookmark.sty}{\usepackage{bookmark}}{\usepackage{hyperref}}
\hypersetup{
  pdftitle={Movielens Movie Recommendation System A Harvard Capstone Project},
  pdfauthor={Manoj Bijoor},
  hidelinks,
  pdfcreator={LaTeX via pandoc}}
\urlstyle{same} % disable monospaced font for URLs
\usepackage[margin=1in]{geometry}
\usepackage{color}
\usepackage{fancyvrb}
\newcommand{\VerbBar}{|}
\newcommand{\VERB}{\Verb[commandchars=\\\{\}]}
\DefineVerbatimEnvironment{Highlighting}{Verbatim}{commandchars=\\\{\}}
% Add ',fontsize=\small' for more characters per line
\newenvironment{Shaded}{}{}
\newcommand{\AlertTok}[1]{\textcolor[rgb]{1.00,0.00,0.00}{\textbf{#1}}}
\newcommand{\AnnotationTok}[1]{\textcolor[rgb]{0.38,0.63,0.69}{\textbf{\textit{#1}}}}
\newcommand{\AttributeTok}[1]{\textcolor[rgb]{0.49,0.56,0.16}{#1}}
\newcommand{\BaseNTok}[1]{\textcolor[rgb]{0.25,0.63,0.44}{#1}}
\newcommand{\BuiltInTok}[1]{#1}
\newcommand{\CharTok}[1]{\textcolor[rgb]{0.25,0.44,0.63}{#1}}
\newcommand{\CommentTok}[1]{\textcolor[rgb]{0.38,0.63,0.69}{\textit{#1}}}
\newcommand{\CommentVarTok}[1]{\textcolor[rgb]{0.38,0.63,0.69}{\textbf{\textit{#1}}}}
\newcommand{\ConstantTok}[1]{\textcolor[rgb]{0.53,0.00,0.00}{#1}}
\newcommand{\ControlFlowTok}[1]{\textcolor[rgb]{0.00,0.44,0.13}{\textbf{#1}}}
\newcommand{\DataTypeTok}[1]{\textcolor[rgb]{0.56,0.13,0.00}{#1}}
\newcommand{\DecValTok}[1]{\textcolor[rgb]{0.25,0.63,0.44}{#1}}
\newcommand{\DocumentationTok}[1]{\textcolor[rgb]{0.73,0.13,0.13}{\textit{#1}}}
\newcommand{\ErrorTok}[1]{\textcolor[rgb]{1.00,0.00,0.00}{\textbf{#1}}}
\newcommand{\ExtensionTok}[1]{#1}
\newcommand{\FloatTok}[1]{\textcolor[rgb]{0.25,0.63,0.44}{#1}}
\newcommand{\FunctionTok}[1]{\textcolor[rgb]{0.02,0.16,0.49}{#1}}
\newcommand{\ImportTok}[1]{#1}
\newcommand{\InformationTok}[1]{\textcolor[rgb]{0.38,0.63,0.69}{\textbf{\textit{#1}}}}
\newcommand{\KeywordTok}[1]{\textcolor[rgb]{0.00,0.44,0.13}{\textbf{#1}}}
\newcommand{\NormalTok}[1]{#1}
\newcommand{\OperatorTok}[1]{\textcolor[rgb]{0.40,0.40,0.40}{#1}}
\newcommand{\OtherTok}[1]{\textcolor[rgb]{0.00,0.44,0.13}{#1}}
\newcommand{\PreprocessorTok}[1]{\textcolor[rgb]{0.74,0.48,0.00}{#1}}
\newcommand{\RegionMarkerTok}[1]{#1}
\newcommand{\SpecialCharTok}[1]{\textcolor[rgb]{0.25,0.44,0.63}{#1}}
\newcommand{\SpecialStringTok}[1]{\textcolor[rgb]{0.73,0.40,0.53}{#1}}
\newcommand{\StringTok}[1]{\textcolor[rgb]{0.25,0.44,0.63}{#1}}
\newcommand{\VariableTok}[1]{\textcolor[rgb]{0.10,0.09,0.49}{#1}}
\newcommand{\VerbatimStringTok}[1]{\textcolor[rgb]{0.25,0.44,0.63}{#1}}
\newcommand{\WarningTok}[1]{\textcolor[rgb]{0.38,0.63,0.69}{\textbf{\textit{#1}}}}
\usepackage{graphicx}
\makeatletter
\def\maxwidth{\ifdim\Gin@nat@width>\linewidth\linewidth\else\Gin@nat@width\fi}
\def\maxheight{\ifdim\Gin@nat@height>\textheight\textheight\else\Gin@nat@height\fi}
\makeatother
% Scale images if necessary, so that they will not overflow the page
% margins by default, and it is still possible to overwrite the defaults
% using explicit options in \includegraphics[width, height, ...]{}
\setkeys{Gin}{width=\maxwidth,height=\maxheight,keepaspectratio}
% Set default figure placement to htbp
\makeatletter
\def\fps@figure{htbp}
\makeatother
% Make links footnotes instead of hotlinks:
\DeclareRobustCommand{\href}[2]{#2\footnote{\url{#1}}}
\setlength{\emergencystretch}{3em} % prevent overfull lines
\providecommand{\tightlist}{%
  \setlength{\itemsep}{0pt}\setlength{\parskip}{0pt}}
\setcounter{secnumdepth}{5}
\usepackage[utf8]{inputenc} \usepackage[english]{babel} \usepackage{bookmark} \usepackage[]{hyperref} \hypersetup{ backref, pdftitle={"Movielens Harvard Capstone"}, bookmarks=true, bookmarksnumbered=true, bookmarksopen=true, bookmarksopenlevel=3, pdfpagemode=FullScreen, pdfstartpage=1, hyperindex=true, pageanchor=true, colorlinks=true, linkcolor=blue, filecolor=magenta, urlcolor=cyan } \usepackage{amsmath} \usepackage{pdflscape} \usepackage[titles]{tocloft} \usepackage{tocloft} \usepackage{titlesec} \usepackage{longtable} \usepackage{xpatch} \usepackage[T1]{fontenc} \usepackage{imakeidx} \makeindex[columns=3, title=Alphabetical Index, intoc]
\usepackage{subfig}
\usepackage{booktabs}
\usepackage{longtable}
\usepackage{array}
\usepackage{multirow}
\usepackage{wrapfig}
\usepackage{float}
\usepackage{colortbl}
\usepackage{pdflscape}
\usepackage{tabu}
\usepackage{threeparttable}
\usepackage{threeparttablex}
\usepackage[normalem]{ulem}
\usepackage{makecell}
\usepackage{xcolor}
\ifLuaTeX
  \usepackage{selnolig}  % disable illegal ligatures
\fi

\title{Movielens\\
Movie Recommendation System\\
A Harvard Capstone Project}
\author{Manoj Bijoor}
\date{March 18, 2021}

\begin{document}
\maketitle

\bookmark[dest=TitlePage]{Title Page}

\pagenumbering{roman}

\newpage

\newpage

\begin{center}

\hypertarget{Abstract}{}
\large{Abstract}
\bookmark[dest=Abstract]{Abstract}

\end{center}

\bigskip

\ldots this is the abstract text\ldots{}

\newpage 
\clearpage
\phantomsection
\setcounter{secnumdepth}{5}
\setcounter{tocdepth}{5}

\cleardoublepage  \hypertarget{toc}{}
\bookmark[dest=toc,level=chapter]{\contentsname} \tableofcontents

\clearpage

\newpage
\clearpage
\phantomsection

\hypertarget{list-of-tables}{%
\section*{List of tables}\label{list-of-tables}}
\addcontentsline{toc}{section}{List of tables}

\renewcommand{\listtablename}{}

\listoftables
\clearpage

\newpage
\clearpage
\phantomsection

\hypertarget{list-of-figures}{%
\section*{List of figures}\label{list-of-figures}}
\addcontentsline{toc}{section}{List of figures}

\renewcommand{\listfigurename}{}

\listoffigures
\clearpage

\newpage
\clearpage
\phantomsection
\newcommand{\listequationsname}{List of Equations}
\newlistof{equations}{equ}{List of Equations}
\newcommand{\equations}[1]{%
\refstepcounter{equations}
\addcontentsline{equ}{equations}{ \protect\numberline{\theequations}#1}\par}
\xpretocmd{\listofequations}{\addcontentsline{toc}{section}{List of Equations}}{}{}

\renewcommand{\listequationsname}{}

\listofequations
\clearpage

\newpage

\pagenumbering{arabic}

\hypertarget{project-overview-movielens---a-harvard-capstone-project}{%
\section{Project Overview: MovieLens - A Harvard Capstone
Project}\label{project-overview-movielens---a-harvard-capstone-project}}

A movie recommendation system using the MovieLens dataset.

For this project, I will be creating a movie recommendation system using
the MovieLens dataset, provided by
\href{https://grouplens.org/}{GroupLens Research}, a research lab in the
Department of Computer Science and Engineering at the University of
Minnesota, Twin Cities specializing in recommender systems, online
communities, mobile and ubiquitous technologies, digital libraries, and
local geographic information systems.

\href{https://grouplens.org/datasets/movielens/}{GroupLens Research} has
collected and made available rating data sets from the
\href{https://movielens.org}{MovieLens web site}. The data sets were
collected over various periods of time, depending on the size of the
set.

I will use the \href{https://grouplens.org/datasets/movielens/10m/}{10M
version of the MovieLens dataset} to make the computation a little
easier.

\textbf{First}, I will download the MovieLens data and run code provided
to generate my datasets.

\textbf{Second}, I will train a machine learning algorithm using the
inputs in one subset to predict movie ratings in the validation set.

\hypertarget{create-train-and-final-hold-out-test-sets}{%
\subsection{Create Train and Final Hold-out Test
Sets}\label{create-train-and-final-hold-out-test-sets}}

I will develop my algorithm using the edx set. For a final test of my
final algorithm, I predict movie ratings in the validation set (the
final hold-out test set) as if they were unknown.
\href{https://en.wikipedia.org/wiki/Root-mean-square_deviation}{RMSE}
will be used to evaluate how close my predictions are to the true values
in the validation set (the final hold-out test set). My target is RMSE
\textless{} 0.86490.

\hypertarget{important-data-sets-usage}{%
\subsubsection{Important: Data sets
usage}\label{important-data-sets-usage}}

The validation data (the final hold-out test set) will NOT be used for
training, developing, or selecting my algorithm and it will ONLY be used
for evaluating the RMSE of my final algorithm. The final hold-out test
set will only be used at the end of my project with my final model. It
will not be used to test the RMSE of multiple models during model
development. I will split the edx data into separate training and test
sets to design and test my algorithm.

\hypertarget{final-product}{%
\subsection{Final Product}\label{final-product}}

\hypertarget{my-submission-for-this-project-is-three-files}{%
\subsubsection{My submission for this project is three
files:}\label{my-submission-for-this-project-is-three-files}}

\begin{enumerate}
\def\labelenumi{\arabic{enumi}.}
\tightlist
\item
  My report in Rmd format
\item
  My report in PDF format (knit from my Rmd file)
\item
  A script in R format that generates my predicted movie ratings and
  RMSE score (contains all code and comments for my project)
\end{enumerate}

The report documents the analysis and presents the findings, along with
supporting statistics and figures. The report assumes that the reader is
not familiar with the project or the data. The report includes the RMSE
generated and the following sections:

\begin{enumerate}
\def\labelenumi{\arabic{enumi}.}
\item
  an introduction/overview/executive summary section that describes the
  dataset and summarizes the goal of the project and key steps that were
  performed
\item
  a methods/analysis section that explains the process and techniques
  used, including data cleaning, data exploration and visualization,
  insights gained, and my modeling approach
\item
  a results section that presents the modeling results and discusses the
  model performance
\item
  a conclusion section that gives a brief summary of the report, its
  limitations and future work
\end{enumerate}

\newpage

\hypertarget{exploratory-data-analysis}{%
\section{Exploratory Data Analysis}\label{exploratory-data-analysis}}

\hypertarget{data-wrangling}{%
\subsection{Data Wrangling}\label{data-wrangling}}

\href{https://en.wikipedia.org/wiki/Data_wrangling}{Data wrangling},
sometimes referred to as data munging, is the process of transforming
and mapping data from one ``raw'' data form into another format with the
intent of making it more appropriate and valuable for a variety of
downstream purposes such as analytics.

The main steps could be described as follows:\\
1. Discovering\\
2. Structuring\\
3. Cleaning\\
4. Enriching\\
5. Validating\\
6. Publishing

Let's perform the steps or combinations thereof starting with Initial
data Exploration \& Visualization in the next few subsections.

\hypertarget{initial-data-exploration-visualization}{%
\subsubsection{Initial data Exploration \&
Visualization}\label{initial-data-exploration-visualization}}

The first ten rows out of 9000055 rows of the Movielens data can be
found in Table \ref{tbl:movielens_data}.

\begin{table}[H]

\caption{\label{tab:eda_1}Movielens data\label{tbl:movielens_data}}
\centering
\fontsize{7}{9}\selectfont
\begin{tabular}[t]{rrrrll}
\toprule
userId & movieId & rating & timestamp & title & genres\\
\midrule
1 & 122 & 5 & 838985046 & Boomerang (1992) & Comedy|Romance\\
1 & 185 & 5 & 838983525 & Net, The (1995) & Action|Crime|Thriller\\
1 & 292 & 5 & 838983421 & Outbreak (1995) & Action|Drama|Sci-Fi|Thriller\\
1 & 316 & 5 & 838983392 & Stargate (1994) & Action|Adventure|Sci-Fi\\
1 & 329 & 5 & 838983392 & Star Trek: Generations (1994) & Action|Adventure|Drama|Sci-Fi\\
1 & 355 & 5 & 838984474 & Flintstones, The (1994) & Children|Comedy|Fantasy\\
1 & 356 & 5 & 838983653 & Forrest Gump (1994) & Comedy|Drama|Romance|War\\
1 & 362 & 5 & 838984885 & Jungle Book, The (1994) & Adventure|Children|Romance\\
1 & 364 & 5 & 838983707 & Lion King, The (1994) & Adventure|Animation|Children|Drama|Musical\\
1 & 370 & 5 & 838984596 & Naked Gun 33 1/3: The Final Insult (1994) & Action|Comedy\\
\bottomrule
\end{tabular}
\end{table}

Each row represents a rating given by one user to one movie.

We can see the number of unique users that provided ratings and how many
unique movies were rated. The unique genres here is based on a column
called ``genres'' that includes every genre that applies to the movie.
Some movies fall under several genres. Defining a category as whatever
combination appears in this column, we are going to refer to this
category as ``unique genres'' or simply ``genres'' from now on.

The number of unique users, movies and genres can be found in Table
\ref{tbl:uniq_users_movies_genres}.

\begin{table}[H]

\caption{\label{tab:eda_2}Unique Users, Movies and Genres\label{tbl:uniq_users_movies_genres}}
\centering
\begin{tabular}[t]{rrr}
\toprule
unique\_users & unique\_movies & unique\_genres\\
\midrule
69878 & 10677 & 797\\
\bottomrule
\end{tabular}
\end{table}

If we multiply the number of unique users by number of unique movies, we
get a very large number, actually 746087406, yet our data table has
9000055 rows. This implies that not every user rated every movie. So we
can think of these data as a very large matrix, with users on the rows
and movies on the columns, with many empty cells. The `gather' function
permits us to convert it to this format, but if we try it for the entire
matrix, it will crash R.

Let's show the matrix of seven users ie: userId's 13-20 and four movies
in Table \ref{tbl:matrix_seven_users_four_movies}.

\begin{table}[H]

\caption{\label{tab:eda_3}Matrix of seven users and four movies\label{tbl:matrix_seven_users_four_movies}}
\centering
\fontsize{8}{10}\selectfont
\begin{tabular}[t]{rrrrr}
\toprule
userId & Forrest Gump (1994) & Jurassic Park (1993) & Pulp Fiction (1994) & Silence of the Lambs, The (1991)\\
\midrule
13 & NA & NA & 4 & NA\\
16 & NA & 3 & NA & NA\\
17 & NA & NA & NA & 5\\
18 & NA & 3 & 5 & 5\\
19 & 4 & 1 & NA & NA\\
\bottomrule
\end{tabular}
\end{table}

\newpage

\hypertarget{a-very-sparse-matrix}{%
\paragraph{A Very Sparse Matrix}\label{a-very-sparse-matrix}}

You can think of the task of a recommendation system as filling in the
`NAs' in the table above. To see how sparse the matrix is, here is the
matrix in Figure \ref{fig:matr} for a random sample of 100 movies and
100 users with yellow indicating a user/movie combination for which we
have a rating.

\begin{figure}
\centering
\includegraphics{figures/eda_4-1.pdf}
\caption{A Very Sparse Matrix\label{fig:matr}}
\end{figure}

This machine learning challenge is quite complicated, because each
outcome \(Y\) has a different set of predictors. To see this, note that
if we are predicting the rating for movie \emph{i} by user \emph{u}, in
principle, all other ratings related to movie \emph{i} and by user
\emph{u} may be used as predictors, but different users rate different
movies and a different number of movies. Furthermore, we may be able to
use information from other movies that we have determined are similar to
movie \emph{i} or from users determined to be similar to user \emph{u}.
In essence, the entire matrix can be used as predictors for each cell.

\newpage

\hypertarget{general-properties-of-the-data}{%
\paragraph{General Properties of the
data}\label{general-properties-of-the-data}}

Let's look at some of the \textbf{\emph{general properties}} of the data
to better understand the challenges.

The \textbf{\emph{first thing}} we notice is that some movies get rated
more than others. Figure \ref{fig:movies_getting_rated} shows the Movies
getting rated distribution. This should not surprise us given that there
are blockbuster movies watched by millions and artsy, independent movies
watched by just a few:

\begin{figure}
\centering
\includegraphics{figures/eda_5-1.pdf}
\caption{Movies getting rated
distribution\label{fig:movies_getting_rated}}
\end{figure}

\newpage

Our \textbf{\emph{second observation}} is that some users are more
active than others at rating movies. Figure
\ref{fig:users_rating_movies} shows Users rating movies distribution:

\begin{figure}
\centering
\includegraphics{figures/eda_6-1.pdf}
\caption{Users rating movies
distribution\label{fig:users_rating_movies}}
\end{figure}

\newpage

\hypertarget{further-data-exploration-visualization-modification}{%
\subsubsection{Further data Exploration, Visualization \&
Modification}\label{further-data-exploration-visualization-modification}}

\hypertarget{modify_edx}{%
\paragraph{Modify edx}\label{modify_edx}}

\textbf{Convert timestamp} in Movielens edx data Table
\ref{tbl:movielens_data} into date-time, a more readable and useful
format named \emph{rating\_date} in Table
\ref{tbl:movielens_edx_data_rating_date} below.

\begin{table}[H]

\caption{\label{tab:dw_1}Movielens edx data with rating date-time\label{tbl:movielens_edx_data_rating_date}}
\centering
\fontsize{8}{10}\selectfont
\begin{tabular}[t]{rrrlll}
\toprule
userId & movieId & rating & title & genres & rating\_date\\
\midrule
1 & 122 & 5 & Boomerang (1992) & Comedy|Romance & 1996-08-02 11:24:06\\
1 & 185 & 5 & Net, The (1995) & Action|Crime|Thriller & 1996-08-02 10:58:45\\
1 & 292 & 5 & Outbreak (1995) & Action|Drama|Sci-Fi|Thriller & 1996-08-02 10:57:01\\
1 & 316 & 5 & Stargate (1994) & Action|Adventure|Sci-Fi & 1996-08-02 10:56:32\\
1 & 329 & 5 & Star Trek: Generations (1994) & Action|Adventure|Drama|Sci-Fi & 1996-08-02 10:56:32\\
\bottomrule
\end{tabular}
\end{table}

\textbf{Split title} in Movielens edx data Table
\ref{tbl:movielens_data} into title and year movie released, a more
useful format named \emph{movie\_dt} in Table
\ref{tbl:movielens_edx_movie_release_date} below.

\begin{table}[H]

\caption{\label{tab:dw_2}Movielens edx data with movie release date\label{tbl:movielens_edx_movie_release_date}}
\centering
\fontsize{8}{10}\selectfont
\begin{tabular}[t]{rrrlllr}
\toprule
userId & movieId & rating & title & genres & rating\_date & movie\_dt\\
\midrule
1 & 122 & 5 & Boomerang & Comedy|Romance & 1996-08-02 11:24:06 & 1992\\
1 & 185 & 5 & Net, The & Action|Crime|Thriller & 1996-08-02 10:58:45 & 1995\\
1 & 292 & 5 & Outbreak & Action|Drama|Sci-Fi|Thriller & 1996-08-02 10:57:01 & 1995\\
1 & 316 & 5 & Stargate & Action|Adventure|Sci-Fi & 1996-08-02 10:56:32 & 1994\\
1 & 329 & 5 & Star Trek: Generations & Action|Adventure|Drama|Sci-Fi & 1996-08-02 10:56:32 & 1994\\
\bottomrule
\end{tabular}
\end{table}

\newpage

\hypertarget{modify-validation-repeat-above-steps}{%
\paragraph{Modify validation, repeat above
steps}\label{modify-validation-repeat-above-steps}}

\textbf{Convert timestamp} in Movielens validation data Table
\ref{tbl:movielens_data} into date-time, a more readable and useful
format named \emph{rating\_date} in Table
\ref{tbl:movielens_val_data_rating_date} below.

\begin{table}[H]

\caption{\label{tab:dw_3}Movielens validation data with rating date-time\label{tbl:movielens_val_data_rating_date}}
\centering
\fontsize{8}{10}\selectfont
\begin{tabular}[t]{rrrlll}
\toprule
userId & movieId & rating & title & genres & rating\_date\\
\midrule
1 & 231 & 5 & Dumb \& Dumber (1994) & Comedy & 1996-08-02 10:56:32\\
1 & 480 & 5 & Jurassic Park (1993) & Action|Adventure|Sci-Fi|Thriller & 1996-08-02 11:00:53\\
1 & 586 & 5 & Home Alone (1990) & Children|Comedy & 1996-08-02 11:07:48\\
2 & 151 & 3 & Rob Roy (1995) & Action|Drama|Romance|War & 1997-07-07 03:34:10\\
2 & 858 & 2 & Godfather, The (1972) & Crime|Drama & 1997-07-07 03:20:45\\
\bottomrule
\end{tabular}
\end{table}

\textbf{Split title} in Movielens validation data Table
\ref{tbl:movielens_data} into title and year movie released, a more
useful format named \emph{movie\_dt} in Table
\ref{tbl:movielens_val_movie_release_date} below.

\begin{table}[H]

\caption{\label{tab:dw_4}Movielens validation data with movie release date\label{tbl:movielens_val_movie_release_date}}
\centering
\fontsize{8}{10}\selectfont
\begin{tabular}[t]{rrrlllr}
\toprule
userId & movieId & rating & title & genres & rating\_date & movie\_dt\\
\midrule
1 & 231 & 5 & Dumb \& Dumber & Comedy & 1996-08-02 10:56:32 & 1994\\
1 & 480 & 5 & Jurassic Park & Action|Adventure|Sci-Fi|Thriller & 1996-08-02 11:00:53 & 1993\\
1 & 586 & 5 & Home Alone & Children|Comedy & 1996-08-02 11:07:48 & 1990\\
2 & 151 & 3 & Rob Roy & Action|Drama|Romance|War & 1997-07-07 03:34:10 & 1995\\
2 & 858 & 2 & Godfather, The & Crime|Drama & 1997-07-07 03:20:45 & 1972\\
\bottomrule
\end{tabular}
\end{table}

\newpage

\hypertarget{genres-combinations-per-movie---a-closer-look}{%
\paragraph{Genres combinations per movie - a closer
look}\label{genres-combinations-per-movie---a-closer-look}}

The movielens data Table \ref{tbl:movielens_edx_avg_rating_time_effect}
also has a genres column. This column includes every genre that applies
to the movie. Some movies fall under several genres. We define a
category of genres as whatever combination of genres appears in this
column, and refer to it as simply ``genres''.

Here we keep only categories with more than 1,000 ratings. Then compute
the average and standard error for each category, and plot these as
error bar plots. See Figure \ref{fig:genres_error_bar_plots} :

\begin{figure}
\centering
\includegraphics{figures/gnr-1.pdf}
\caption{Movies genres error bar
plots\label{fig:genres_error_bar_plots}}
\end{figure}

The plot shows strong evidence of a genre effect.

\newpage

\hypertarget{movie-rating-date-time---a-closer-look}{%
\paragraph{Movie Rating Date-Time - a closer
look}\label{movie-rating-date-time---a-closer-look}}

The Movielens edx data Table \ref{tbl:movielens_data} also includes a
time stamp. This variable represents the time and date in which the
rating was provided. The units are seconds since January 1, 1970. We
create a new column date with the date named \emph{rating\_date} in
subsection \protect\hyperlink{modify_edx}{Modify edx} to get Table
\ref{tbl:movielens_edx_movie_release_date} .

We compute the average rating for each week and plot this average
against day. See Figure
\ref{fig:movies_average_ratings_for_each_week_versus_day} :

\begin{verbatim}
`geom_smooth()` using method = 'loess' and formula 'y ~ x'
\end{verbatim}

\begin{figure}
\centering
\includegraphics{figures/rd_1-1.pdf}
\caption{Movies average ratings for each week versus
day\label{fig:movies_average_ratings_for_each_week_versus_day}}
\end{figure}

The plot shows some evidence of a time effect. If we define \(d_{u,i}\)
as the day for user's \emph{u} rating of movie \emph{i}, then the
following model given by Equation \ref{eq:EqTimeEffect} is most
appropriate:

%
\refstepcounter{equations}
\addcontentsline{equ}{equations}{ \protect\numberline{\theequations}Movie Rating Date-Time effect Equation \ref{eq:EqTimeEffect}}\par

\label{eq:EqTimeEffect} \begin{equation}
Y_{u,i} = \mu + b_{i} + b_{u}  + f(d_{u,i})+ \epsilon_{u,i}\text{, with f a smooth function of }d_{u,i}
\end{equation}

\textbf{\emph{Modify edx}} Let's update the \textbf{\emph{edx}} table
with a new column for the average rating for each week and another
column for the day rounded to the nearest value of the week to get Table
\ref{tbl:movielens_edx_avg_rating_time_effect} below:

\begin{table}[H]

\caption{\label{tab:rd_2}Movielens edx data with average rating due to rating time effect\label{tbl:movielens_edx_avg_rating_time_effect}}
\centering
\fontsize{6}{8}\selectfont
\begin{tabular}[t]{rrrlllrlr}
\toprule
userId & movieId & rating & title & genres & rating\_date & movie\_dt & date & avg\_rating\\
\midrule
1 & 122 & 5 & Boomerang & Comedy|Romance & 1996-08-02 11:24:06 & 1992 & 1996-08-04 & 3.538801\\
1 & 185 & 5 & Net, The & Action|Crime|Thriller & 1996-08-02 10:58:45 & 1995 & 1996-08-04 & 3.538801\\
1 & 292 & 5 & Outbreak & Action|Drama|Sci-Fi|Thriller & 1996-08-02 10:57:01 & 1995 & 1996-08-04 & 3.538801\\
1 & 316 & 5 & Stargate & Action|Adventure|Sci-Fi & 1996-08-02 10:56:32 & 1994 & 1996-08-04 & 3.538801\\
1 & 329 & 5 & Star Trek: Generations & Action|Adventure|Drama|Sci-Fi & 1996-08-02 10:56:32 & 1994 & 1996-08-04 & 3.538801\\
\bottomrule
\end{tabular}
\end{table}

\textbf{TODO: Repeat above for validation data as well and somehow add
this to the modelling section}

\textbf{\emph{Modify validation}} We need to do the above
\textbf{\emph{avg\_rating\_time\_effect}} update for the validation data
as well. Let's update the \textbf{\emph{validation}} table to get Table
\ref{tbl:movielens_validation_avg_rating_time_effect} below:

\begin{table}[H]

\caption{\label{tab:rd_3}Movielens validation data with average rating due to rating time effect\label{tbl:movielens_validation_avg_rating_time_effect}}
\centering
\fontsize{6}{8}\selectfont
\begin{tabular}[t]{rrrlllrlr}
\toprule
userId & movieId & rating & title & genres & rating\_date & movie\_dt & date & avg\_rating\\
\midrule
1 & 231 & 5 & Dumb \& Dumber & Comedy & 1996-08-02 10:56:32 & 1994 & 1996-08-04 & 3.555820\\
1 & 480 & 5 & Jurassic Park & Action|Adventure|Sci-Fi|Thriller & 1996-08-02 11:00:53 & 1993 & 1996-08-04 & 3.555820\\
1 & 586 & 5 & Home Alone & Children|Comedy & 1996-08-02 11:07:48 & 1990 & 1996-08-04 & 3.555820\\
2 & 151 & 3 & Rob Roy & Action|Drama|Romance|War & 1997-07-07 03:34:10 & 1995 & 1997-07-06 & 3.606571\\
2 & 858 & 2 & Godfather, The & Crime|Drama & 1997-07-07 03:20:45 & 1972 & 1997-07-06 & 3.606571\\
\bottomrule
\end{tabular}
\end{table}

\newpage

\hypertarget{movie-release-date---a-closer-look}{%
\paragraph{Movie Release Date - a closer
look}\label{movie-release-date---a-closer-look}}

Computing the number of ratings for each movie and then plotting it
against the year the movie came out, that is the release date and using
the square root transformation on the counts using Table
\ref{tbl:movielens_edx_movie_release_date} , we get see Figure
\ref{fig:ratings_movie_release_date_all_dates} :

\textbf{TODO: Align images}

\begin{figure}[h!]

{\centering \subfloat[All data points only\label{fig:md_1-1}]{\includegraphics[width=0.7\linewidth]{figures/md_1-1} }\newline\subfloat[Smooth line through all data points\label{fig:md_1-2}]{\includegraphics[width=0.7\linewidth]{figures/md_1-2} }

}

\caption{Ratings Movie Release Date - All dates\label{fig:ratings_movie_release_date_all_dates}}\label{fig:md_1}
\end{figure}

we see that, on average, movies that came out after 1993 get more
ratings. We also see that with newer movies, starting in 1993, the
number of ratings decreases with year: the more recent a movie is, the
less time users have had to rate it.

\newpage

Among movies that came out in 1993 or later, we select the top 25 movies
with the highest average number of ratings per year (n/year) and
calculate the average rating of each of them. To calculate number of
ratings per year, use 2018 as the end year. See Figure
\ref{fig:25_movies_avg_and_most_ratings_per_year_post_1993} :

\begin{verbatim}
`geom_smooth()` using method = 'loess' and formula 'y ~ x'
\end{verbatim}

\begin{figure}
\centering
\includegraphics{figures/md_3-1.pdf}
\caption{25 Movies with the most ratings per year and their average
rating post
1993\label{fig:25_movies_avg_and_most_ratings_per_year_post_1993}}
\end{figure}

\newpage

We see that the most rated movies tend to have above average ratings.
This is not surprising: more people watch popular movies. To confirm
this, we stratify the post 1993 movies by ratings per year and compute
their average ratings. Figure
\ref{fig:movies_average_ratings_versus_ratings_per_year_post_1993} is a
plot of average ratings versus ratings per year showing an estimate of
the trend.

We see that the more a movie is rated, the higher the rating.

\textbf{Post-1993 movies}

\begin{verbatim}
`geom_smooth()` using method = 'gam' and formula 'y ~ s(x, bs = "cs")'
\end{verbatim}

\begin{figure}
\centering
\includegraphics{figures/md_4-1.pdf}
\caption{Movies average ratings versus ratings per year post
1993\label{fig:movies_average_ratings_versus_ratings_per_year_post_1993}}
\end{figure}

\newpage

\textbf{Pre-1993 movies}\\
Compare Pre-1993 movies trend shown here in Figure
\ref{fig:movies_average_ratings_versus_ratings_per_year_pre_1993} Versus
Post-1993 movies trend in Figure
\ref{fig:movies_average_ratings_versus_ratings_per_year_post_1993}
above.

\begin{verbatim}
`geom_smooth()` using method = 'gam' and formula 'y ~ s(x, bs = "cs")'
\end{verbatim}

\begin{figure}
\centering
\includegraphics{figures/md_5-1.pdf}
\caption{Movies average ratings versus ratings per year pre
1993\label{fig:movies_average_ratings_versus_ratings_per_year_pre_1993}}
\end{figure}

\newpage

\textbf{\emph{Modify edx data for Release Date Effect}} We stratify the
movies by ratings per year and compute their average ratings based on
what we learnt above, where we confirmed our intuition that more people
watch popular movies. Finally edx data table looks as shown in Table
\ref{tbl:movielens_edx_avg_release_date_effect} below. Figure
\ref{fig:movies_average_ratings_versus_ratings_per_year_for_all_years_for_edx}"\}
is a plot of average ratings versus ratings per year showing an estimate
of the trend.

\textbf{\emph{All Years}}

\begin{verbatim}
`geom_smooth()` using method = 'gam' and formula 'y ~ s(x, bs = "cs")'
\end{verbatim}

\begin{figure}
\centering
\includegraphics{figures/md_6-1.pdf}
\caption{Movies average ratings versus ratings per year for all years
for
edx\label{fig:movies_average_ratings_versus_ratings_per_year_for_all_years_for_edx}}
\end{figure}

\begin{table}[H]

\caption{\label{tab:md_7}Movielens edx data with average rating due to release date effect\label{tbl:movielens_edx_avg_release_date_effect}}
\centering
\fontsize{4}{6}\selectfont
\begin{tabular}[t]{rrrlllrlrrrrr}
\toprule
userId & movieId & rating & title & genres & rating\_date & movie\_dt & date & avg\_rating & avg\_rating\_rel & n & years & n\_year\\
\midrule
1 & 122 & 5 & Boomerang & Comedy|Romance & 1996-08-02 11:24:06 & 1992 & 1996-08-04 & 3.538801 & 2.858586 & 2178 & 26 & 84\\
1 & 185 & 5 & Net, The & Action|Crime|Thriller & 1996-08-02 10:58:45 & 1995 & 1996-08-04 & 3.538801 & 3.129334 & 13469 & 23 & 586\\
1 & 292 & 5 & Outbreak & Action|Drama|Sci-Fi|Thriller & 1996-08-02 10:57:01 & 1995 & 1996-08-04 & 3.538801 & 3.418011 & 14447 & 23 & 628\\
1 & 316 & 5 & Stargate & Action|Adventure|Sci-Fi & 1996-08-02 10:56:32 & 1994 & 1996-08-04 & 3.538801 & 3.349677 & 17030 & 24 & 710\\
1 & 329 & 5 & Star Trek: Generations & Action|Adventure|Drama|Sci-Fi & 1996-08-02 10:56:32 & 1994 & 1996-08-04 & 3.538801 & 3.337457 & 14550 & 24 & 606\\
\bottomrule
\end{tabular}
\end{table}

\begin{verbatim}
tibble [9,000,055 x 13] (S3: tbl_df/tbl/data.frame)
 $ userId        : int [1:9000055] 1 1 1 1 1 1 1 1 1 1 ...
 $ movieId       : num [1:9000055] 122 185 292 316 329 355 356 362 364 370 ...
 $ rating        : num [1:9000055] 5 5 5 5 5 5 5 5 5 5 ...
 $ title         : chr [1:9000055] "Boomerang " "Net, The " "Outbreak " "Stargate " ...
 $ genres        : chr [1:9000055] "Comedy|Romance" "Action|Crime|Thriller" "Action|Drama|Sci-Fi|Thriller" "Action|Adventure|Sci-Fi" ...
 $ rating_date   : POSIXct[1:9000055], format: "1996-08-02 11:24:06" "1996-08-02 10:58:45" ...
 $ movie_dt      : num [1:9000055] 1992 1995 1995 1994 1994 ...
 $ date          : POSIXct[1:9000055], format: "1996-08-04" "1996-08-04" ...
 $ avg_rating    : num [1:9000055] 3.54 3.54 3.54 3.54 3.54 ...
 $ avg_rating_rel: num [1:9000055] 2.86 3.13 3.42 3.35 3.34 ...
 $ n             : int [1:9000055] 2178 13469 14447 17030 14550 4831 31079 3612 18921 7331 ...
 $ years         : num [1:9000055] 26 23 23 24 24 24 24 24 24 24 ...
 $ n_year        : num [1:9000055] 84 586 628 710 606 ...
\end{verbatim}

\newpage

\textbf{\emph{Modify validation data for Release Date Effect}} We need
to do the above \textbf{\emph{avg\_rating\_rel\_effect}} update for the
validation data as well. Let's update the \textbf{\emph{validation}}
table to get Table
\ref{tbl:movielens_validation_avg_release_date_effect} below:

\textbf{\emph{All Years}}

\begin{table}[H]

\caption{\label{tab:md_8}Movielens validation data with average rating due to release date effect\label{tbl:movielens_validation_avg_release_date_effect}}
\centering
\fontsize{4}{6}\selectfont
\begin{tabular}[t]{rrrlllrlrrrrr}
\toprule
userId & movieId & rating & title & genres & rating\_date & movie\_dt & date & avg\_rating & avg\_rating\_rel & n & years & n\_year\\
\midrule
1 & 231 & 5 & Dumb \& Dumber & Comedy & 1996-08-02 10:56:32 & 1994 & 1996-08-04 & 3.555820 & 2.953281 & 1798 & 24 & 75\\
1 & 480 & 5 & Jurassic Park & Action|Adventure|Sci-Fi|Thriller & 1996-08-02 11:00:53 & 1993 & 1996-08-04 & 3.555820 & 3.643993 & 3271 & 25 & 131\\
1 & 586 & 5 & Home Alone & Children|Comedy & 1996-08-02 11:07:48 & 1990 & 1996-08-04 & 3.555820 & 3.074550 & 1556 & 28 & 56\\
2 & 151 & 3 & Rob Roy & Action|Drama|Romance|War & 1997-07-07 03:34:10 & 1995 & 1997-07-06 & 3.606571 & 3.571984 & 771 & 23 & 34\\
2 & 858 & 2 & Godfather, The & Crime|Drama & 1997-07-07 03:20:45 & 1972 & 1997-07-06 & 3.606571 & 4.412675 & 2067 & 46 & 45\\
\bottomrule
\end{tabular}
\end{table}

\begin{verbatim}
tibble [999,999 x 13] (S3: tbl_df/tbl/data.frame)
 $ userId        : int [1:999999] 1 1 1 2 2 2 3 3 4 4 ...
 $ movieId       : num [1:999999] 231 480 586 151 858 ...
 $ rating        : num [1:999999] 5 5 5 3 2 3 3.5 4.5 5 3 ...
 $ title         : chr [1:999999] "Dumb & Dumber " "Jurassic Park " "Home Alone " "Rob Roy " ...
 $ genres        : chr [1:999999] "Comedy" "Action|Adventure|Sci-Fi|Thriller" "Children|Comedy" "Action|Drama|Romance|War" ...
 $ rating_date   : POSIXct[1:999999], format: "1996-08-02 10:56:32" "1996-08-02 11:00:53" ...
 $ movie_dt      : num [1:999999] 1994 1993 1990 1995 1972 ...
 $ date          : POSIXct[1:999999], format: "1996-08-04" "1996-08-04" ...
 $ avg_rating    : num [1:999999] 3.56 3.56 3.56 3.61 3.61 ...
 $ avg_rating_rel: num [1:999999] 2.95 3.64 3.07 3.57 4.41 ...
 $ n             : int [1:999999] 1798 3271 1556 771 2067 862 2545 947 1869 776 ...
 $ years         : num [1:999999] 24 25 28 23 46 21 28 17 23 24 ...
 $ n_year        : num [1:999999] 75 131 56 34 45 41 91 56 81 32 ...
\end{verbatim}

\newpage

\hypertarget{analysis---model-building-and-evaluation}{%
\section{Analysis - Model Building and
Evaluation}\label{analysis---model-building-and-evaluation}}

\hypertarget{split-the-edx-data-into-separate-training-and-test-sets}{%
\subsection{Split the edx data into separate training and test
sets}\label{split-the-edx-data-into-separate-training-and-test-sets}}

We will develop our algorithm using the edx set only.\\
We will split the edx data into separate training and test sets to
design and test our algorithm, namely train\_set and test\_set.

\hypertarget{loss-function}{%
\subsubsection{Loss function}\label{loss-function}}

For a final test of our algorithm, we predict movie ratings in the test
set as if they were unknown.
\href{https://en.wikipedia.org/wiki/Root-mean-square_deviation}{RMSE}
(residual mean squared error/root mean square error), the typical error
loss, will be used to evaluate how close our predictions are to the true
values in the validation set.\\
We define \({y_{u,i}}\) as the rating for movie \emph{i} by user
\emph{u} and denote our prediction with \({\hat{y}_{u,i}}\).

The RMSE is then defined as Equation \ref{eq:EqLossFunction}:

%
\refstepcounter{equations}
\addcontentsline{equ}{equations}{ \protect\numberline{\theequations}Loss function RMSE Equation \ref{eq:EqLossFunction}}\par

\label{eq:EqLossFunction} \begin{equation}
  RMSE=\sqrt{\frac{1}{N}\sum_{u,i}(\hat{y}_{u,i}-y_{u,i})^{2}}
\end{equation}

with N being the number of user/movie combinations and the sum occurring
over all these combinations.

Remember that we can interpret the RMSE similarly to a standard
deviation: it is the typical error we make when predicting a movie
rating. If this number is larger than 1, it means our typical error is
larger than one star, which is not good.

Let's write a function that computes the RMSE for vectors of ratings and
their corresponding predictors:

\begin{Shaded}
\begin{Highlighting}[]
\NormalTok{RMSE }\OtherTok{\textless{}{-}} \ControlFlowTok{function}\NormalTok{(true\_ratings, predicted\_ratings) \{}
    \FunctionTok{sqrt}\NormalTok{(}\FunctionTok{mean}\NormalTok{((true\_ratings }\SpecialCharTok{{-}}\NormalTok{ predicted\_ratings)}\SpecialCharTok{\^{}}\DecValTok{2}\NormalTok{))}
\NormalTok{\}}
\end{Highlighting}
\end{Shaded}

\newpage

\hypertarget{model-1-a-first-naive-mean-model}{%
\subsection{Model 1: A first naive ``mean''
model}\label{model-1-a-first-naive-mean-model}}

Let's start by building the simplest possible recommendation system: we
predict the same rating for all movies regardless of user. A model that
assumes the same rating for all movies and users with all the
differences explained by random variation would look like Equation
\ref{eq:EqModel1}:

%
\refstepcounter{equations}
\addcontentsline{equ}{equations}{ \protect\numberline{\theequations}Model 1: A naive "mean" model Equation \ref{eq:EqModel1}}\par

\label{eq:EqModel1} \begin{equation}
  Y_{i,i} = \mu + \epsilon_{u,i}
\end{equation}

with \(\epsilon_{u,i}\) independent errors sampled from the same
distribution centered at 0 and \(\mu\) the ``true'' rating for all
movies. We know that the estimate that minimizes the RMSE is the least
squares estimate of \(\mu\) and, in this case, is the average of all
ratings:

\begin{Shaded}
\begin{Highlighting}[]
\NormalTok{(mu\_hat }\OtherTok{\textless{}{-}} \FunctionTok{mean}\NormalTok{(train\_set}\SpecialCharTok{$}\NormalTok{rating))}
\NormalTok{[}\DecValTok{1}\NormalTok{] }\FloatTok{3.512482}
\end{Highlighting}
\end{Shaded}

If we predict all unknown ratings with \(\hat\mu\) we obtain the
following RMSE:

\begin{Shaded}
\begin{Highlighting}[]
\NormalTok{(model\_1\_rmse }\OtherTok{\textless{}{-}} \FunctionTok{RMSE}\NormalTok{(test\_set}\SpecialCharTok{$}\NormalTok{rating, mu\_hat))}
\NormalTok{[}\DecValTok{1}\NormalTok{] }\FloatTok{1.059904}
\end{Highlighting}
\end{Shaded}

Keep in mind that if we plug in any other number, we get a higher RMSE.
For example:

\begin{Shaded}
\begin{Highlighting}[]
\NormalTok{predictions }\OtherTok{\textless{}{-}} \FunctionTok{rep}\NormalTok{(}\FloatTok{2.5}\NormalTok{, }\FunctionTok{nrow}\NormalTok{(test\_set))}
\FunctionTok{RMSE}\NormalTok{(test\_set}\SpecialCharTok{$}\NormalTok{rating, predictions)}
\NormalTok{[}\DecValTok{1}\NormalTok{] }\FloatTok{1.465736}

\NormalTok{predictions }\OtherTok{\textless{}{-}} \FunctionTok{rep}\NormalTok{(}\DecValTok{3}\NormalTok{, }\FunctionTok{nrow}\NormalTok{(test\_set))}
\FunctionTok{RMSE}\NormalTok{(test\_set}\SpecialCharTok{$}\NormalTok{rating, predictions)}
\NormalTok{[}\DecValTok{1}\NormalTok{] }\FloatTok{1.177271}

\NormalTok{predictions }\OtherTok{\textless{}{-}} \FunctionTok{rep}\NormalTok{(}\DecValTok{4}\NormalTok{, }\FunctionTok{nrow}\NormalTok{(test\_set))}
\FunctionTok{RMSE}\NormalTok{(test\_set}\SpecialCharTok{$}\NormalTok{rating, predictions)}
\NormalTok{[}\DecValTok{1}\NormalTok{] }\FloatTok{1.166678}
\end{Highlighting}
\end{Shaded}

From looking at the distribution of ratings, we can visualize that this
is the standard deviation of that distribution. We get a RMSE of about
1. Our target is RMSE \textless{} 0.86490. So we can definitely do
better!

\newpage

\hypertarget{results-model-1}{%
\subsubsection{Results Model 1}\label{results-model-1}}

As we go along, we will be comparing different approaches. Let's start
by creating a results table with this naive approach to get Table
\ref{tbl:rmse_results_model_1}:

\begin{table}[H]

\caption{\label{tab:m_1_4}RMSE Results Model 1\label{tbl:rmse_results_model_1}}
\centering
\fontsize{7}{9}\selectfont
\begin{tabular}[t]{llr}
\toprule
Index & Method & RMSE\\
\midrule
1 & Just the average & 1.059904\\
\bottomrule
\end{tabular}
\end{table}

\newpage

\hypertarget{model-2-movie-effects}{%
\subsection{Model 2: Movie effects}\label{model-2-movie-effects}}

We know from experience that some movies are just generally rated higher
than others. This intuition, that different movies are rated
differently, is confirmed by data. We can augment our previous model by
adding the term \(b_{i}\) to represent average ranking for movie
\emph{i} and would look like Equation \ref{eq:EqModel2-1}:

%
\refstepcounter{equations}
\addcontentsline{equ}{equations}{ \protect\numberline{\theequations}Model 2: Movie effects linear model Equation \ref{eq:EqModel2-1}}\par

\label{eq:EqModel2-1} \begin{equation}
  Y_{u,i} = \mu + b_{i} + \epsilon_{u,i}
\end{equation}

Statistics textbooks refer to the \emph{b}s as effects or
\emph{``bias''}.

We can again use least squares to estimate the \(b_{i}\) in the
following way to get Equation \ref{eq:EqModel2-2}:

%
\refstepcounter{equations}
\addcontentsline{equ}{equations}{ \protect\numberline{\theequations}LSE linear function to fit Movie effects linear model Equation \ref{eq:EqModel2-2}}\par

\label{eq:EqModel2-2} \begin{equation}
  fit \leftarrow lm(rating \; \sim \; as.factor(userId), \: data = train\_{}set)
\end{equation}

Because there are thousands of \(b_{i}\) as each movie gets one, the
lm() function will be very slow here. We therefore will not run the code
above.

But in this particular situation, we know that the least squares
estimate \(\hat{b_{i}}\) is just the average of \(Y_{u,i}-\hat{\mu}\)
for each movie \emph{i}. \textbf{\emph{So we can compute them this way
(we will drop the hat notation in the code to represent estimates going
forward)}}:

%
\refstepcounter{equations}
\addcontentsline{equ}{equations}{ \protect\numberline{\theequations}Movie specific effects Equation \ref{eq:EqModel2-3}}\par

\label{eq:EqModel2-3} \begin{equation}
  \hat{b_{i}} = \overline{y_{u,i} - \hat{\mu}}
\end{equation}

\begin{Shaded}
\begin{Highlighting}[]
\NormalTok{mu }\OtherTok{\textless{}{-}} \FunctionTok{mean}\NormalTok{(train\_set}\SpecialCharTok{$}\NormalTok{rating)}
\NormalTok{movie\_avgs }\OtherTok{\textless{}{-}}\NormalTok{ train\_set }\SpecialCharTok{\%\textgreater{}\%} \FunctionTok{group\_by}\NormalTok{(movieId) }\SpecialCharTok{\%\textgreater{}\%} \FunctionTok{summarize}\NormalTok{(}\AttributeTok{b\_i =} \FunctionTok{mean}\NormalTok{(rating }\SpecialCharTok{{-}} 
\NormalTok{    mu))}
\end{Highlighting}
\end{Shaded}

\newpage

We can see that these estimates vary substantially, see Figure
\ref{fig:model_2}

\begin{figure}
\centering
\includegraphics{figures/m_2_3-1.pdf}
\caption{Movie effect or bias distribution\label{fig:model_2}}
\end{figure}

Remember \(\hat{\mu}\)=3.5 so a \(b_{i}\)=1.5 implies a perfect five
star rating. Let's see how much our prediction improves once we use
\(\hat{y_{u,i}}=\hat{\mu}+\hat{b_{i}}\):

\begin{Shaded}
\begin{Highlighting}[]
\NormalTok{predicted\_ratings\_model\_2 }\OtherTok{\textless{}{-}}\NormalTok{ mu }\SpecialCharTok{+}\NormalTok{ test\_set }\SpecialCharTok{\%\textgreater{}\%} \FunctionTok{left\_join}\NormalTok{(movie\_avgs, }
    \AttributeTok{by =} \StringTok{"movieId"}\NormalTok{) }\SpecialCharTok{\%\textgreater{}\%}\NormalTok{ .}\SpecialCharTok{$}\NormalTok{b\_i}
\NormalTok{(model\_2\_rmse }\OtherTok{\textless{}{-}} \FunctionTok{RMSE}\NormalTok{(predicted\_ratings\_model\_2, test\_set}\SpecialCharTok{$}\NormalTok{rating))}
\NormalTok{[}\DecValTok{1}\NormalTok{] }\FloatTok{0.9437429}
\end{Highlighting}
\end{Shaded}

\newpage

\hypertarget{results-model-1-2}{%
\subsubsection{Results Model 1-2}\label{results-model-1-2}}

Let's add the movie effects model to our results table to get Table
\ref{tbl:rmse_results_model_1-2}

\begin{table}[H]

\caption{\label{tab:m_2_5}RMSE Results Model 1-2\label{tbl:rmse_results_model_1-2}}
\centering
\fontsize{7}{9}\selectfont
\begin{tabular}[t]{llr}
\toprule
Index & Method & RMSE\\
\midrule
1 & Just the average & 1.0599043\\
2 & Movie Effect Model & 0.9437429\\
\bottomrule
\end{tabular}
\end{table}

\newpage

\hypertarget{model-3-user-effects}{%
\subsection{Model 3: User effects}\label{model-3-user-effects}}

Let's compute \emph{b\_u} the average rating for user \emph{u} for those
that have rated over 100 movies, see Figure
\ref{fig:average_ratings_for_users_who_have_rated_over_100_movies}

\begin{Shaded}
\begin{Highlighting}[]
\NormalTok{train\_set }\SpecialCharTok{\%\textgreater{}\%} \FunctionTok{group\_by}\NormalTok{(userId) }\SpecialCharTok{\%\textgreater{}\%} \FunctionTok{summarize}\NormalTok{(}\AttributeTok{b\_u =} \FunctionTok{mean}\NormalTok{(rating)) }\SpecialCharTok{\%\textgreater{}\%} 
    \FunctionTok{filter}\NormalTok{(}\FunctionTok{n}\NormalTok{() }\SpecialCharTok{\textgreater{}=} \DecValTok{100}\NormalTok{) }\SpecialCharTok{\%\textgreater{}\%} \FunctionTok{ggplot}\NormalTok{(}\FunctionTok{aes}\NormalTok{(b\_u)) }\SpecialCharTok{+} \FunctionTok{geom\_histogram}\NormalTok{(}\AttributeTok{bins =} \DecValTok{30}\NormalTok{, }
    \AttributeTok{color =} \StringTok{"black"}\NormalTok{) }\SpecialCharTok{+} \FunctionTok{ggtitle}\NormalTok{(}\StringTok{"Average rating for users who have rated over 100 movies"}\NormalTok{)}
\end{Highlighting}
\end{Shaded}

\begin{figure}
\centering
\includegraphics{figures/ue_1-1.pdf}
\caption{Average rating for users who have rated over 100
movies\label{fig:average_ratings_for_users_who_have_rated_over_100_movies}}
\end{figure}

\newpage

Let's compute \emph{b\_u} the average rating for user \emph{u} for those
that have rated any movies, see Figure
\ref{fig:average_ratings_for_users_who_have_rated_any_movies}

\begin{Shaded}
\begin{Highlighting}[]
\NormalTok{train\_set }\SpecialCharTok{\%\textgreater{}\%} \FunctionTok{group\_by}\NormalTok{(userId) }\SpecialCharTok{\%\textgreater{}\%} \FunctionTok{summarize}\NormalTok{(}\AttributeTok{b\_u =} \FunctionTok{mean}\NormalTok{(rating)) }\SpecialCharTok{\%\textgreater{}\%} 
    \FunctionTok{ggplot}\NormalTok{(}\FunctionTok{aes}\NormalTok{(b\_u)) }\SpecialCharTok{+} \FunctionTok{geom\_histogram}\NormalTok{(}\AttributeTok{bins =} \DecValTok{30}\NormalTok{, }\AttributeTok{color =} \StringTok{"black"}\NormalTok{) }\SpecialCharTok{+} 
    \FunctionTok{ggtitle}\NormalTok{(}\StringTok{"Average rating for users who have rated any movies"}\NormalTok{)}
\end{Highlighting}
\end{Shaded}

\begin{figure}
\centering
\includegraphics{figures/ue_2-1.pdf}
\caption{Average rating for users who have rated any
movies\label{fig:average_ratings_for_users_who_have_rated_any_movies}}
\end{figure}

Notice that there is substantial variability across users as well: some
users are very cranky and others love every movie. This implies that a
further improvement to our model may be as shown in Equation
\ref{eq:EqModel3-1}:

%
\refstepcounter{equations}
\addcontentsline{equ}{equations}{ \protect\numberline{\theequations}Model 3: Movie + User effects linear model Equation \ref{eq:EqModel3-1}}\par

\label{eq:EqModel3-1} \begin{equation}
  Y_{u,i} = \mu + b_{i} + b_{u} + \epsilon_{u,i}
\end{equation}

where \(b_{u}\) is a user-specific effect. Now if a cranky user
(negative \(b_{u}\)) rates a great movie (positive \(b_{i}\)), the
effects counter each other and we may be able to correctly predict that
this user gave this great movie a 3 rather than a 5.

To fit this model, we could again use lm() as shown in Equation
\ref{eq:EqModel3-2}:

%
\refstepcounter{equations}
\addcontentsline{equ}{equations}{ \protect\numberline{\theequations}LSE linear function to fit Movie + User effects linear model Equation \ref{eq:EqModel3-2}}\par

\label{eq:EqModel3-2} \begin{equation}
  fit \leftarrow lm(rating \; \sim \; as.factor(movieId) + as.factor(userId), \: data = train\_{}set)
\end{equation}

but, for the reasons described earlier, we won't. Instead, we will
compute an approximation by computing \(\hat{\mu}\) and \(\hat{b_{i}}\)
and estimating \(\hat{b_{u}}\) as the average of
\(y_{u,i}-\hat{\mu}-\hat{b_{i}}\):

%
\refstepcounter{equations}
\addcontentsline{equ}{equations}{ \protect\numberline{\theequations}User specific effects Equation \ref{eq:EqModel3-3}}\par

\label{eq:EqModel3-3} \begin{equation}
  \hat{b_{u}} = \overline{y_{u,i} - \hat{\mu} - \hat{b_{i}}}
\end{equation}

\newpage

\begin{Shaded}
\begin{Highlighting}[]
\NormalTok{user\_avgs }\OtherTok{\textless{}{-}}\NormalTok{ train\_set }\SpecialCharTok{\%\textgreater{}\%} \FunctionTok{left\_join}\NormalTok{(movie\_avgs, }\AttributeTok{by =} \StringTok{"movieId"}\NormalTok{) }\SpecialCharTok{\%\textgreater{}\%} 
    \FunctionTok{group\_by}\NormalTok{(userId) }\SpecialCharTok{\%\textgreater{}\%} \FunctionTok{summarize}\NormalTok{(}\AttributeTok{b\_u =} \FunctionTok{mean}\NormalTok{(rating }\SpecialCharTok{{-}}\NormalTok{ mu }\SpecialCharTok{{-}}\NormalTok{ b\_i))}
\end{Highlighting}
\end{Shaded}

We can see that these estimates vary substantially, see Figure
\ref{fig:model_3}

\begin{figure}
\centering
\includegraphics{figures/ue_4-1.pdf}
\caption{User effect or bias distribution\label{fig:model_3}}
\end{figure}

We can now construct predictors and see how much the RMSE improves:

\begin{Shaded}
\begin{Highlighting}[]
\NormalTok{predicted\_ratings\_model\_3 }\OtherTok{\textless{}{-}}\NormalTok{ test\_set }\SpecialCharTok{\%\textgreater{}\%} \FunctionTok{left\_join}\NormalTok{(movie\_avgs, }
    \AttributeTok{by =} \StringTok{"movieId"}\NormalTok{) }\SpecialCharTok{\%\textgreater{}\%} \FunctionTok{left\_join}\NormalTok{(user\_avgs, }\AttributeTok{by =} \StringTok{"userId"}\NormalTok{) }\SpecialCharTok{\%\textgreater{}\%} 
    \FunctionTok{mutate}\NormalTok{(}\AttributeTok{pred =}\NormalTok{ mu }\SpecialCharTok{+}\NormalTok{ b\_i }\SpecialCharTok{+}\NormalTok{ b\_u) }\SpecialCharTok{\%\textgreater{}\%}\NormalTok{ .}\SpecialCharTok{$}\NormalTok{pred}
\NormalTok{(model\_3\_rmse }\OtherTok{\textless{}{-}} \FunctionTok{RMSE}\NormalTok{(predicted\_ratings\_model\_3, test\_set}\SpecialCharTok{$}\NormalTok{rating))}
\NormalTok{[}\DecValTok{1}\NormalTok{] }\FloatTok{0.865932}
\end{Highlighting}
\end{Shaded}

\newpage

\hypertarget{results-table-model-1-3}{%
\subsubsection{Results Table Model 1-3}\label{results-table-model-1-3}}

Let's add the user effects model to our results table to get Table
\ref{tbl:rmse_results_model_1-3}

\begin{table}[H]

\caption{\label{tab:ue_6}RMSE Results Model 1-3\label{tbl:rmse_results_model_1-3}}
\centering
\fontsize{7}{9}\selectfont
\begin{tabular}[t]{llr}
\toprule
Index & Method & RMSE\\
\midrule
1 & Just the average & 1.0599043\\
2 & Movie Effect Model & 0.9437429\\
3 & Movie + User Effects Model & 0.8659320\\
\bottomrule
\end{tabular}
\end{table}

\newpage

\hypertarget{model-4-genre-effects}{%
\subsection{Model 4: Genre effects}\label{model-4-genre-effects}}

The movielens data also has a genres column. This column includes every
genre that applies to the movie. Some movies fall under several genres.
Define a category of genres as whatever combination of genres appears in
this column. We will refer to this category as simply ``genres''.

There is strong evidence of a genre effect as we have shown earlier in
Figure \ref{fig:genres_error_bar_plots}, and in this section below in
Figure \ref{fig:model_4}. If we define \(g_{u,i}\) as the genre for
\emph{u} user's rating of movie \emph{i}, then the following model as
shown in Equation \ref{eq:EqModel4-1} is most appropriate:

%
\refstepcounter{equations}
\addcontentsline{equ}{equations}{ \protect\numberline{\theequations}Model 4: Movie + User + Genre effects linear model Equation \ref{eq:EqModel4-1}}\par

\label{eq:EqModel4-1} \begin{equation}
  Y_{u,i} = \mu + b_{i} + b_{u} + \sum_{k=1}^Kx_{u,i}\beta_k + \epsilon_{u,i}
\end{equation}

\begin{center}
with $x_{u,i}^k=1$ if $g_{u,i}$ is genre *k*
\end{center}

\begin{Shaded}
\begin{Highlighting}[]
\NormalTok{train\_set }\SpecialCharTok{\%\textgreater{}\%} \FunctionTok{group\_by}\NormalTok{(genres) }\SpecialCharTok{\%\textgreater{}\%} \FunctionTok{summarize}\NormalTok{(}\AttributeTok{mu\_g =} \FunctionTok{mean}\NormalTok{(rating)) }\SpecialCharTok{\%\textgreater{}\%} 
    \FunctionTok{ggplot}\NormalTok{(}\FunctionTok{aes}\NormalTok{(mu\_g)) }\SpecialCharTok{+} \FunctionTok{geom\_histogram}\NormalTok{(}\AttributeTok{bins =} \DecValTok{30}\NormalTok{, }\AttributeTok{color =} \StringTok{"black"}\NormalTok{) }\SpecialCharTok{+} 
    \FunctionTok{ggtitle}\NormalTok{(}\StringTok{"Average rating for movies of category genres"}\NormalTok{)}
\end{Highlighting}
\end{Shaded}

\begin{figure}
\centering
\includegraphics{figures/ge_1-1.pdf}
\caption{Average rating for movies of category
genres\label{fig:average_ratings_for_movies_of_category_genres}}
\end{figure}

To fit this model, we could again use the lm() function as shown in
Equation \ref{eq:EqModel4-2}:

%
\refstepcounter{equations}
\addcontentsline{equ}{equations}{ \protect\numberline{\theequations}LSE linear function to fit Movie + User + Genres effects linear model Equation \ref{eq:EqModel4-2}}\par

\label{eq:EqModel4-2} \begin{equation}
\begin{split}
  fit \leftarrow lm(rating \; \sim \; & as.factor(movieId) + as.factor(userId) + \\ 
  & as.factor(genres), \: data = train\_{}set)
\end{split}
\end{equation}

but, for the reasons described earlier, we won't. Instead, we will
compute an approximation by computing \(\hat{\mu}\), \(\hat{b_{i}}\),
\(\hat{b_{u}}\) and estimating \(\hat{b_{g}}\) as the average of
\(y_{u,i}-\hat{\mu}-\hat{b_{i}}-\hat{b_{u}}\) :

%
\refstepcounter{equations}
\addcontentsline{equ}{equations}{ \protect\numberline{\theequations}Genres specific effects Equation \ref{eq:EqModel4-3}}\par

\label{eq:EqModel4-3} \begin{equation}
  \hat{b_{g}} = \overline{y_{u,i} - \hat{\mu} - \hat{b_{i}} - \hat{b_{u}}}
\end{equation}

where:

%
\refstepcounter{equations}
\addcontentsline{equ}{equations}{ \protect\numberline{\theequations}Genres specific effects Equation \ref{eq:EqModel4-4}}\par

\label{eq:EqModel4-4} \begin{equation}
  \hat{b_{g}} = \sum_{k=1}^Kx_{u,i}\beta_k
\end{equation}

\begin{center}
with $x_{u,i}^k=1$ if $g_{u,i}$ is genre *k*
\end{center}

where \(\hat{b_{g}}\) is genre specific effect.

\begin{Shaded}
\begin{Highlighting}[]
\NormalTok{genres\_avgs }\OtherTok{\textless{}{-}}\NormalTok{ train\_set }\SpecialCharTok{\%\textgreater{}\%} \FunctionTok{left\_join}\NormalTok{(movie\_avgs, }\AttributeTok{by =} \StringTok{"movieId"}\NormalTok{) }\SpecialCharTok{\%\textgreater{}\%} 
    \FunctionTok{left\_join}\NormalTok{(user\_avgs, }\AttributeTok{by =} \StringTok{"userId"}\NormalTok{) }\SpecialCharTok{\%\textgreater{}\%} \FunctionTok{group\_by}\NormalTok{(genres) }\SpecialCharTok{\%\textgreater{}\%} 
    \FunctionTok{summarize}\NormalTok{(}\AttributeTok{b\_g =} \FunctionTok{mean}\NormalTok{(rating }\SpecialCharTok{{-}}\NormalTok{ mu }\SpecialCharTok{{-}}\NormalTok{ b\_i }\SpecialCharTok{{-}}\NormalTok{ b\_u))}
\end{Highlighting}
\end{Shaded}

We can see that these estimates vary substantially, see Figure
\ref{fig:model_4}

\begin{figure}
\centering
\includegraphics{figures/ge_5-1.pdf}
\caption{Genres effect or bias distribution\label{fig:model_4}}
\end{figure}

We can now construct predictors and see how much the RMSE improves:

\begin{Shaded}
\begin{Highlighting}[]
\NormalTok{predicted\_ratings\_model\_4 }\OtherTok{\textless{}{-}}\NormalTok{ test\_set }\SpecialCharTok{\%\textgreater{}\%} \FunctionTok{left\_join}\NormalTok{(movie\_avgs, }
    \AttributeTok{by =} \StringTok{"movieId"}\NormalTok{) }\SpecialCharTok{\%\textgreater{}\%} \FunctionTok{left\_join}\NormalTok{(user\_avgs, }\AttributeTok{by =} \StringTok{"userId"}\NormalTok{) }\SpecialCharTok{\%\textgreater{}\%} 
    \FunctionTok{left\_join}\NormalTok{(genres\_avgs, }\AttributeTok{by =} \StringTok{"genres"}\NormalTok{) }\SpecialCharTok{\%\textgreater{}\%} \FunctionTok{mutate}\NormalTok{(}\AttributeTok{pred =}\NormalTok{ mu }\SpecialCharTok{+} 
\NormalTok{    b\_i }\SpecialCharTok{+}\NormalTok{ b\_u }\SpecialCharTok{+}\NormalTok{ b\_g) }\SpecialCharTok{\%\textgreater{}\%}\NormalTok{ .}\SpecialCharTok{$}\NormalTok{pred}
\NormalTok{(model\_4\_rmse }\OtherTok{\textless{}{-}} \FunctionTok{RMSE}\NormalTok{(predicted\_ratings\_model\_4, test\_set}\SpecialCharTok{$}\NormalTok{rating))}
\NormalTok{[}\DecValTok{1}\NormalTok{] }\FloatTok{0.8655941}
\end{Highlighting}
\end{Shaded}

\newpage

\hypertarget{results-table-model-1-4}{%
\subsubsection{Results Table Model 1-4}\label{results-table-model-1-4}}

Let's add the genres effects model to our results table to get Table
\ref{tbl:rmse_results_model_1-4}

\begin{table}[H]

\caption{\label{tab:ge_7}RMSE Results Model 1-4\label{tbl:rmse_results_model_1-4}}
\centering
\fontsize{7}{9}\selectfont
\begin{tabular}[t]{llr}
\toprule
Index & Method & RMSE\\
\midrule
1 & Just the average & 1.0599043\\
2 & Movie Effect Model & 0.9437429\\
3 & Movie + User Effects Model & 0.8659320\\
4 & Movie + User + Genres Effects Model & 0.8655941\\
\bottomrule
\end{tabular}
\end{table}

\newpage

\hypertarget{model-5-rating-time-effect}{%
\subsection{Model 5 : Rating Time
effect}\label{model-5-rating-time-effect}}

The movielens dataset also includes a time stamp. This variable
represents the time and date in which the rating was provided. Earlier
in the EDA/Data wrangling section we created a new column date with the
time stamp.

We computed the average rating for each week and plotted this average
against day.

The plot shows some evidence of a time effect. If we define \(d_{u,i}\)
as the day for user's \emph{u} rating of movie \emph{i}, then the
following updated model as shown in Equation \ref{eq:EqModel5-1} is most
appropriate:\\

%
\refstepcounter{equations}
\addcontentsline{equ}{equations}{ \protect\numberline{\theequations}Model 5: Movie + User + Genre + Rating time effects linear model Equation \ref{eq:EqModel5-1}}\par

\label{eq:EqModel5-1} \begin{equation}
  Y_{u,i} = \mu + b_{i} + b_{u} + \sum_{k=1}^Kx_{u,i}\beta_k + f(d_{u,i}) + \epsilon_{u,i}
\end{equation}

\begin{center}
with f a smooth function of $d_{u,i}$
\end{center}

To fit this model, we could again use lm() function as shown in Equation
\ref{eq:EqModel5-2}:

%
\refstepcounter{equations}
\addcontentsline{equ}{equations}{ \protect\numberline{\theequations}LSE linear function to fit Movie + User + Genres + Rating time effects linear model Equation \ref{eq:EqModel5-2}}\par

\label{eq:EqModel5-2} \begin{equation}
\begin{split}
  fit \leftarrow lm(rating \; \sim \; & as.factor(movieId) + as.factor(userId) + \\ 
  & as.factor(genres) + as.factor(date), \: data = train\_{}set)
\end{split}
\end{equation}

but, for the reasons described earlier, we won't. Instead, we will
compute an approximation by computing \(\hat{\mu}\), \(\hat{b_{i}}\),
\(\hat{b_{u}}\), \(\hat{b_{g}}\) and estimating \(\hat{b_{d}}\) as the
average of \(y_{u,i}-\hat{\mu}-\hat{b_{i}}-\hat{b_{u}}-\hat{b_{g}}\) :\\

%
\refstepcounter{equations}
\addcontentsline{equ}{equations}{ \protect\numberline{\theequations}Rating time specific effects Equation \ref{eq:EqModel5-3}}\par

\label{eq:EqModel5-3} \begin{equation}
  \hat{b_{d}} = \overline{y_{u,i} - \hat{\mu} - \hat{b_{i}} - \hat{b_{u}}  - \hat{b_{u}}}
\end{equation}

where:

%
\refstepcounter{equations}
\addcontentsline{equ}{equations}{ \protect\numberline{\theequations}Rating time specific effects in function form Equation \ref{eq:EqModel5-4}}\par

\label{eq:EqModel5-4} \begin{equation}
  \hat{b_{d}} = f(d_{u,i})
\end{equation}

where \(\hat{b_{d}}\) is rating time specific effect.

\begin{Shaded}
\begin{Highlighting}[]
\NormalTok{time\_effect\_avgs }\OtherTok{\textless{}{-}}\NormalTok{ train\_set }\SpecialCharTok{\%\textgreater{}\%} \FunctionTok{left\_join}\NormalTok{(movie\_avgs, }\AttributeTok{by =} \StringTok{"movieId"}\NormalTok{) }\SpecialCharTok{\%\textgreater{}\%} 
    \FunctionTok{left\_join}\NormalTok{(user\_avgs, }\AttributeTok{by =} \StringTok{"userId"}\NormalTok{) }\SpecialCharTok{\%\textgreater{}\%} \FunctionTok{left\_join}\NormalTok{(genres\_avgs, }
    \AttributeTok{by =} \StringTok{"genres"}\NormalTok{) }\SpecialCharTok{\%\textgreater{}\%} \FunctionTok{group\_by}\NormalTok{(date) }\SpecialCharTok{\%\textgreater{}\%} \FunctionTok{summarize}\NormalTok{(}\AttributeTok{b\_d =} \FunctionTok{mean}\NormalTok{(avg\_rating }\SpecialCharTok{{-}} 
\NormalTok{    mu }\SpecialCharTok{{-}}\NormalTok{ b\_i }\SpecialCharTok{{-}}\NormalTok{ b\_u }\SpecialCharTok{{-}}\NormalTok{ b\_g))}
\end{Highlighting}
\end{Shaded}

\newpage

We can see that these estimates vary substantially, see Figure
\ref{fig:model_4}

\begin{figure}
\centering
\includegraphics{figures/rte_3-1.pdf}
\caption{Rating time effect or bias distribution\label{fig:model_5}}
\end{figure}

We can now construct predictors and see how much the RMSE improves

\begin{Shaded}
\begin{Highlighting}[]
\NormalTok{predicted\_ratings\_model\_5 }\OtherTok{\textless{}{-}}\NormalTok{ test\_set }\SpecialCharTok{\%\textgreater{}\%} \FunctionTok{left\_join}\NormalTok{(movie\_avgs, }
    \AttributeTok{by =} \StringTok{"movieId"}\NormalTok{) }\SpecialCharTok{\%\textgreater{}\%} \FunctionTok{left\_join}\NormalTok{(user\_avgs, }\AttributeTok{by =} \StringTok{"userId"}\NormalTok{) }\SpecialCharTok{\%\textgreater{}\%} 
    \FunctionTok{left\_join}\NormalTok{(genres\_avgs, }\AttributeTok{by =} \StringTok{"genres"}\NormalTok{) }\SpecialCharTok{\%\textgreater{}\%} \FunctionTok{left\_join}\NormalTok{(time\_effect\_avgs, }
    \AttributeTok{by =} \StringTok{"date"}\NormalTok{) }\SpecialCharTok{\%\textgreater{}\%} \FunctionTok{mutate}\NormalTok{(}\AttributeTok{pred =}\NormalTok{ mu }\SpecialCharTok{+}\NormalTok{ b\_i }\SpecialCharTok{+}\NormalTok{ b\_u }\SpecialCharTok{+}\NormalTok{ b\_g }\SpecialCharTok{+}\NormalTok{ b\_d) }\SpecialCharTok{\%\textgreater{}\%} 
\NormalTok{    .}\SpecialCharTok{$}\NormalTok{pred}
\NormalTok{(model\_5\_rmse }\OtherTok{\textless{}{-}} \FunctionTok{RMSE}\NormalTok{(predicted\_ratings\_model\_5, test\_set}\SpecialCharTok{$}\NormalTok{rating))}
\NormalTok{[}\DecValTok{1}\NormalTok{] }\FloatTok{0.8654205}
\end{Highlighting}
\end{Shaded}

\newpage

\hypertarget{results-table-model-1-5}{%
\subsubsection{Results Table Model 1-5}\label{results-table-model-1-5}}

Let's add the Rating Time effects model to our results table to get
Table \ref{tbl:rmse_results_model_1-5}

\begin{table}[H]

\caption{\label{tab:rte_5}RMSE Results Model 1-5\label{tbl:rmse_results_model_1-5}}
\centering
\fontsize{7}{9}\selectfont
\begin{tabular}[t]{llr}
\toprule
Index & Method & RMSE\\
\midrule
1 & Just the average & 1.0599043\\
2 & Movie Effect Model & 0.9437429\\
3 & Movie + User Effects Model & 0.8659320\\
4 & Movie + User + Genres Effects Model & 0.8655941\\
5 & Movie + User + Genres + Rating Time Effects Model & 0.8654205\\
\bottomrule
\end{tabular}
\end{table}

\newpage

\hypertarget{model-6-release-date-effect}{%
\subsection{Model 6 Release Date
Effect}\label{model-6-release-date-effect}}

The plots in Figures \ref{fig:ratings_movie_release_date_all_dates},
\ref{fig:25_movies_avg_and_most_ratings_per_year_post_1993},
\ref{fig:movies_average_ratings_versus_ratings_per_year_post_1993},
\ref{fig:movies_average_ratings_versus_ratings_per_year_pre_1993} above
shows some evidence of a Release Date effect based on the when the movie
was released and it's popularity given by the mean rating. If we define
\(arr_{r,i,y}\) as the average rating \emph{r=mean(rating)} since
release date \emph{y=n\_year} for movie \emph{i} (in the formula for
plots above), then the following updated model is most appropriate:\\

%
\refstepcounter{equations}
\addcontentsline{equ}{equations}{ \protect\numberline{\theequations}Model 6: Movie + User + Genre + Rating time + Release date effects linear model Equation \ref{eq:EqModel6-1}}\par

\label{eq:EqModel6-1} \begin{equation}
  Y_{u,i} = \mu + b_{i} + b_{u} + \sum_{k=1}^Kx_{u,i}\beta_k + f(d_{u,i}) + f(arr_{r,i,y}) + \epsilon_{u,i}
\end{equation}

\begin{center}
with f a smooth function of $arr_{r,i,y}$
\end{center}

To fit this model, we could again use lm() function as shown in Equation
\ref{eq:EqModel6-2}:

%
\refstepcounter{equations}
\addcontentsline{equ}{equations}{ \protect\numberline{\theequations}LSE linear function to fit Movie + User + Genres + Rating time + Release date effects linear model Equation \ref{eq:EqModel6-2}}\par

\label{eq:EqModel6-2} \begin{equation}
\begin{split}
  fit \leftarrow lm(rating \; \sim \; & as.factor(movieId) + as.factor(userId) + \\ 
  & as.factor(genres) + as.factor(date) + \\ 
  & as.factor(movie_dt), \: data = train\_{}set)
\end{split}
\end{equation}

but, for the reasons described earlier, we won't. Instead, we will
compute an approximation by computing \(\hat{\mu}\), \(\hat{b_{i}}\),
\(\hat{b_{u}}\), \(\hat{b_{g}}\), \(\hat{b_{d}}\) and estimating
\(\hat{b_{r}}\) as the average of
\(y_{u,i}-\hat{\mu}-\hat{b_{i}}-\hat{b_{u}}-\hat{b_{g}}-\hat{b_{d}}\)
where:\\

%
\refstepcounter{equations}
\addcontentsline{equ}{equations}{ \protect\numberline{\theequations}Release date specific effects Equation \ref{eq:EqModel6-3}}\par

\label{eq:EqModel6-3} \begin{equation}
  \hat{b_{r}} = \overline{y_{u,i} - \hat{\mu} - \hat{b_{i}} - \hat{b_{u}}  - \hat{b_{g}} - \hat{b_{d}}}
\end{equation}

where:

%
\refstepcounter{equations}
\addcontentsline{equ}{equations}{ \protect\numberline{\theequations}Release date specific effects in function form Equation \ref{eq:EqModel6-4}}\par

\label{eq:EqModel6-4} \begin{equation}
  \hat{b_{r}} = f(arr_{r,i,y})
\end{equation}

where \(\hat{b_{r}}\) is Release date specific effect.

\begin{Shaded}
\begin{Highlighting}[]
\NormalTok{rel\_effect\_avgs }\OtherTok{\textless{}{-}}\NormalTok{ train\_set }\SpecialCharTok{\%\textgreater{}\%} \FunctionTok{left\_join}\NormalTok{(movie\_avgs, }\AttributeTok{by =} \StringTok{"movieId"}\NormalTok{) }\SpecialCharTok{\%\textgreater{}\%} 
    \FunctionTok{left\_join}\NormalTok{(user\_avgs, }\AttributeTok{by =} \StringTok{"userId"}\NormalTok{) }\SpecialCharTok{\%\textgreater{}\%} \FunctionTok{left\_join}\NormalTok{(genres\_avgs, }
    \AttributeTok{by =} \StringTok{"genres"}\NormalTok{) }\SpecialCharTok{\%\textgreater{}\%} \FunctionTok{left\_join}\NormalTok{(time\_effect\_avgs, }\AttributeTok{by =} \StringTok{"date"}\NormalTok{) }\SpecialCharTok{\%\textgreater{}\%} 
    \FunctionTok{group\_by}\NormalTok{(movieId) }\SpecialCharTok{\%\textgreater{}\%} \FunctionTok{summarize}\NormalTok{(}\AttributeTok{b\_r =} \FunctionTok{mean}\NormalTok{(avg\_rating\_rel }\SpecialCharTok{{-}} 
\NormalTok{    mu }\SpecialCharTok{{-}}\NormalTok{ b\_i }\SpecialCharTok{{-}}\NormalTok{ b\_u }\SpecialCharTok{{-}}\NormalTok{ b\_g }\SpecialCharTok{{-}}\NormalTok{ b\_d))}
\end{Highlighting}
\end{Shaded}

\newpage

We can see that these estimates vary substantially, see Figure
\ref{fig:model_6}

\begin{figure}
\centering
\includegraphics{figures/rde_3-1.pdf}
\caption{Release Date effect or bias distribution\label{fig:model_6}}
\end{figure}

We can now construct predictors and see how much the RMSE improves

\begin{Shaded}
\begin{Highlighting}[]
\NormalTok{predicted\_ratings\_model\_6 }\OtherTok{\textless{}{-}}\NormalTok{ test\_set }\SpecialCharTok{\%\textgreater{}\%} 
  \FunctionTok{left\_join}\NormalTok{(movie\_avgs, }\AttributeTok{by=}\StringTok{\textquotesingle{}movieId\textquotesingle{}}\NormalTok{) }\SpecialCharTok{\%\textgreater{}\%} 
  \FunctionTok{left\_join}\NormalTok{(user\_avgs, }\AttributeTok{by=}\StringTok{\textquotesingle{}userId\textquotesingle{}}\NormalTok{) }\SpecialCharTok{\%\textgreater{}\%} 
  \FunctionTok{left\_join}\NormalTok{(genres\_avgs, }\AttributeTok{by=}\StringTok{\textquotesingle{}genres\textquotesingle{}}\NormalTok{) }\SpecialCharTok{\%\textgreater{}\%} 
  \FunctionTok{left\_join}\NormalTok{(time\_effect\_avgs, }\AttributeTok{by =} \StringTok{"date"}\NormalTok{) }\SpecialCharTok{\%\textgreater{}\%} 
  \FunctionTok{left\_join}\NormalTok{(rel\_effect\_avgs, }\AttributeTok{by=}\StringTok{\textquotesingle{}movieId\textquotesingle{}}\NormalTok{) }\SpecialCharTok{\%\textgreater{}\%}
  \FunctionTok{mutate}\NormalTok{(}\AttributeTok{pred =}\NormalTok{ mu }\SpecialCharTok{+}\NormalTok{ b\_i }\SpecialCharTok{+}\NormalTok{ b\_u }\SpecialCharTok{+}\NormalTok{ b\_g }\SpecialCharTok{+}\NormalTok{ b\_d }\SpecialCharTok{+}\NormalTok{ b\_r) }\SpecialCharTok{\%\textgreater{}\%}
\NormalTok{  .}\SpecialCharTok{$}\NormalTok{pred}

\NormalTok{(model\_6\_rmse }\OtherTok{\textless{}{-}} \FunctionTok{RMSE}\NormalTok{(predicted\_ratings\_model\_6, test\_set}\SpecialCharTok{$}\NormalTok{rating))}
\end{Highlighting}
\end{Shaded}

\begin{verbatim}
[1] 0.863333
\end{verbatim}

\newpage

\hypertarget{results-table}{%
\subsubsection{Results Table}\label{results-table}}

Let's add the Release Date effects model to our results table

\begin{table}[H]

\caption{\label{tab:rde_5}RMSE Results Model 1-6\label{tbl:rmse_results_model_1-6}}
\centering
\fontsize{7}{9}\selectfont
\begin{tabular}[t]{llr}
\toprule
Index & Method & RMSE\\
\midrule
1 & Just the average & 1.0599043\\
2 & Movie Effect Model & 0.9437429\\
3 & Movie + User Effects Model & 0.8659320\\
4 & Movie + User + Genres Effects Model & 0.8655941\\
5 & Movie + User + Genres + Rating Time Effects Model & 0.8654205\\
6 & Movie + User + Genres + Time + Releasedate Effects Model & 0.8633330\\
\bottomrule
\end{tabular}
\end{table}

\begin{Shaded}
\begin{Highlighting}[]
\NormalTok{    knitr}\SpecialCharTok{::}\FunctionTok{knit\_exit}\NormalTok{()}
\end{Highlighting}
\end{Shaded}


\end{document}
